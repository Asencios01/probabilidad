% Options for packages loaded elsewhere
\PassOptionsToPackage{unicode}{hyperref}
\PassOptionsToPackage{hyphens}{url}
\PassOptionsToPackage{dvipsnames,svgnames,x11names}{xcolor}
%
\documentclass[
]{article}
\title{Ejercicios Tema 6 - Variables aleatorias muldidimensionales}
\author{Ricardo Alberich, Juan Gabriel Gomila y Arnau Mir}
\date{Curso de Probabilidad y Variables Aleatorias con R y Python}

\usepackage{amsmath,amssymb}
\usepackage{lmodern}
\usepackage{iftex}
\ifPDFTeX
  \usepackage[T1]{fontenc}
  \usepackage[utf8]{inputenc}
  \usepackage{textcomp} % provide euro and other symbols
\else % if luatex or xetex
  \usepackage{unicode-math}
  \defaultfontfeatures{Scale=MatchLowercase}
  \defaultfontfeatures[\rmfamily]{Ligatures=TeX,Scale=1}
\fi
% Use upquote if available, for straight quotes in verbatim environments
\IfFileExists{upquote.sty}{\usepackage{upquote}}{}
\IfFileExists{microtype.sty}{% use microtype if available
  \usepackage[]{microtype}
  \UseMicrotypeSet[protrusion]{basicmath} % disable protrusion for tt fonts
}{}
\makeatletter
\@ifundefined{KOMAClassName}{% if non-KOMA class
  \IfFileExists{parskip.sty}{%
    \usepackage{parskip}
  }{% else
    \setlength{\parindent}{0pt}
    \setlength{\parskip}{6pt plus 2pt minus 1pt}}
}{% if KOMA class
  \KOMAoptions{parskip=half}}
\makeatother
\usepackage{xcolor}
\IfFileExists{xurl.sty}{\usepackage{xurl}}{} % add URL line breaks if available
\IfFileExists{bookmark.sty}{\usepackage{bookmark}}{\usepackage{hyperref}}
\hypersetup{
  pdftitle={Ejercicios Tema 6 - Variables aleatorias muldidimensionales},
  pdfauthor={Ricardo Alberich, Juan Gabriel Gomila y Arnau Mir},
  colorlinks=true,
  linkcolor={red},
  filecolor={Maroon},
  citecolor={blue},
  urlcolor={blue},
  pdfcreator={LaTeX via pandoc}}
\urlstyle{same} % disable monospaced font for URLs
\usepackage[margin=1in]{geometry}
\usepackage{graphicx}
\makeatletter
\def\maxwidth{\ifdim\Gin@nat@width>\linewidth\linewidth\else\Gin@nat@width\fi}
\def\maxheight{\ifdim\Gin@nat@height>\textheight\textheight\else\Gin@nat@height\fi}
\makeatother
% Scale images if necessary, so that they will not overflow the page
% margins by default, and it is still possible to overwrite the defaults
% using explicit options in \includegraphics[width, height, ...]{}
\setkeys{Gin}{width=\maxwidth,height=\maxheight,keepaspectratio}
% Set default figure placement to htbp
\makeatletter
\def\fps@figure{htbp}
\makeatother
\setlength{\emergencystretch}{3em} % prevent overfull lines
\providecommand{\tightlist}{%
  \setlength{\itemsep}{0pt}\setlength{\parskip}{0pt}}
\setcounter{secnumdepth}{5}
\renewcommand{\contentsname}{Contenidos}
\ifLuaTeX
  \usepackage{selnolig}  % disable illegal ligatures
\fi

\begin{document}
\maketitle

{
\hypersetup{linkcolor=blue}
\setcounter{tocdepth}{2}
\tableofcontents
}
\newcommand\momento{m}
\newcommand{\momentocentral}{\mu}
\newcommand{\FunGenMom}{m}
\newcommand{\FunCar}{\phi}

\hypertarget{variables-aleatorias-multidimensionales-discretas}{%
\section{Variables aleatorias multidimensionales
discretas}\label{variables-aleatorias-multidimensionales-discretas}}

\hypertarget{problema-1}{%
\subsection{Problema 1}\label{problema-1}}

Una urna contiene una bola negra y dos bolas blancas. Se sacan tres
bolas de la urna. Sea la variable \(I_k\) que vale 1 si el resultado de
la extracción \(k\)-ésima es la bola negra y vale 0 en caso contrario.
Definimos las siguientes tres variables aleatorias: \[
\begin{array}{rl}
X & = I_1+I_2+I_3, \\
Y & = \min\{I_1,I_2,I_3\},\\
Z & = \max\{I_1,I_2,I_3\}.
\end{array}
\] 1. Especificar el rango de valores de la variable \(3\) dimensional
\((X,Y,Z)\) si las extracciones son con reposición. Hallar la función de
probabilidad conjunta \(P_{XYZ}\). 1. ¿Son las variables \(X\), \(Y\) y
\(Z\) independientes? ¿Son las variables \(X\) e \(Y\) independientes?
1. Repetir el primer apartado suponiendo ahora que las extracciones son
sin reposición.

\hypertarget{soluciuxf3n}{%
\subsubsection{Solución}\label{soluciuxf3n}}

Indicaciones para la solución:

Para resolver este ejercicio, os ayudará mucho ver el ejemplo que tenéis
en los apuntes.

En primer lugar, para cada variable X, Y o Z, hay que calcular el
conjunto de valores que alcanza cada variable.

Daremos los detalles para la variable X. Para las demás, hay que razonar
de forma parecida. Para la variable X, el rango de valores sería el
conjunto sería: \(\{0,1,2,3\}\) ya que pueden salir desde 3 bolas
blancas en cuyo caso la suma sería mínima, 0 hasta 3 bolas negras, en
cuyo caso la suma sería máxima, 3. Ahora para cada para valor entre 0 y
3, hay que estudiar cuantas tripletas \((i,j,k)\) dan como suma dicho
valor, calcular la probabilidad para cada tripleta y sumar todas las
probabilidades.

Para ver si \(X\),\(Y\) y \(Z\) son independientes, hay que hallar en
primer lugar las marginales de cada una de ellas y ver para cada
triplete de valores \((x,y,z)\) si
\(P_{XYZ}(x,y,z)=P_X(x)\cdot P_Y(y)\cdot P_Z(z)\). Si falla en uno, ya
no serían independientes.

Para ver si \(X\) e \(Y\) son independientes hay que ver para cada par
de valores \((x,y)\) si \(P_{XY}(x,y)=P_X(x)\cdot P_Y(y)\), calculando
previamente \(P_{X Y}\), la función de probabilidad conjunta de la
variable \((X,Y)\).

Para resolver el problema pero suponiendo que las extracciones son sin
reposición, hay que tener en cuenta que

\begin{enumerate}
\def\labelenumi{\arabic{enumi}.}
\tightlist
\item
  hay casos que no se pueden dar como por ejemplo \((1,1,0)\) ya que si
  sale una bola negra, y no la volvemos a poner en la urna ya no pueden
  salir más bolas negras y,
\item
  las probabilidades también cambian respecto a las extracciones con
  reposición.
\end{enumerate}

\hypertarget{problema-2.}{%
\subsection{Problema 2.}\label{problema-2.}}

Sean \(X_1, X_2,\ldots, X_n\) variables binarias aleatorias que toman
valores 0 o 1 para indicar si un altavoz está en silencio (0) o activo
(1). Si un altavoz está en silencio, permanece inactivo en el siguiente
intervalo de tiempo con probabilidad 3/4, y un altavoz activo permanece
activo con probabilidad 1/2. Hallar la función de probabilidad conjunta
\(P_{X_1X_2X_3}\) y la función de probabilidad marginal de \(X_3\).
Suponga que el altavoz empieza en el estado silencioso.

\hypertarget{soluciuxf3n-1}{%
\subsubsection{Solución}\label{soluciuxf3n-1}}

Indicaciones para la resolución:

Para resolver este problema, os puede ser de gran ayuda hacer un árbol
binario cuya raíz sea el valor 0 (estado silencioso) y para cada nodo,
indicar que sus hijos pueden ser 0 (estado silencioso) o 1 (estado
activo). Escribir en cada arista la probabilidad que ocurra cada cosa.
Al final escribir el conjunto de resultados posibles junto con su
probabilidad. Por ejemplo, el resultado (0,0,0) tendría probabilidad
\((3/4)\cdot (3/4)\cdot (3/4)=27/64\) y el resultado \((1,1,0)\) tendría
probabilidad \((1/4)\cdot (1/2)\cdot (1/2)=1/16\).

De esta forma, podéis calcular \(P_{X1 X_2 X_3}\).

\hypertarget{problema-3.}{%
\subsection{Problema 3.}\label{problema-3.}}

Un experimento aleatorio tiene cuatro resultados posibles. Supongamos
que el experimento se repite \(n\) veces de forma independiente y sea
\(X_k\) el número de veces que se produce el resultado \(k\)-ésimo. La
función de probabilidad conjunta de la variable \(3\)-dimensional
\((X_1,X_2,X_3)\), \(P_{X_1X_2X_3}\) es la siguiente:

\[
P_{X_1 X_2 X_3}(k_1,k_2,k_3)=\frac{n!\cdot 3!}{(n+3)!}=\binom{n+3}{3}^{-1}, \ \mbox{para }k_i\geq 0,\mbox{ y }k_1+k_2+k_3\leq n.
\]

\begin{enumerate}
\def\labelenumi{\arabic{enumi}.}
\tightlist
\item
  Hallar la función de probabilidad marginal de la variable
  bidimensional \((X_1,X_2)\).
\item
  Hallar la función de probabilidad marginal de la variable \(X_1\).
\item
  Hallar la función de probabilidad condicional de la variable
  \((X_2,X_3)\) dado \(X_1=m\), para \(0\leq m\leq n\).
\end{enumerate}

\hypertarget{soluciuxf3n-2}{%
\subsubsection{Solución}\label{soluciuxf3n-2}}

Indicaciones para la resolución:

\textbf{Este ejercicio tiene nivel muy avanzado.}

Para hallar la función de probabilidad marginal para \((X_1,X_2)\), hay
que hallar en primer lugar el conjunto de valores de la variable
\((X_1,X_2)\) y fijado \((k_1,k_2)\) dentro de dicho conjunto de valores
para hallar \(P_{X_1 X_2}(k_1,k_2)\) hay que sumar los valores
\(P_{X_1 X_2 X_3}(k_1,k_2,k_3)\) para todos los \(k_3\) tal que
\((k_1,k_2,k_3)\) esté en el conjunto de valores de la variable
3-dimensional \((X_1,X_2,X_3)\).

Para la marginal \(X_1\), una vez hallado el conjunto de valores de la
variable \(X_1\) y fijado \(k_1\) en dicho conjunto, para hallar
\(P_{X_1}(k_1)\) hay que sumar todos los valores
\(P_{X_1 X_2 X_3}(k_1,k_2,k_3)\) para todos los \((k_2,k_3)\) tal que
\((k_1,k_2,k_3)\) esté en el conjunto de valores de la variable
3-dimensional \((X_1,X_2,X_3)\).

Para hallar la distribución de la variable condicional de \((X_2,X_3)\)
dado \(X_1=m\), hay en un valor \((k_2,k_3)\) hay que usar la expresión
\(P_{X_1 X_2 X_3}(m,k_2,k_3)/P_{X_1}(m)\).

\hypertarget{variables-aleatorias-multidimensionales-continuas}{%
\section{Variables aleatorias multidimensionales
continuas}\label{variables-aleatorias-multidimensionales-continuas}}

\hypertarget{problema-4}{%
\subsection{Problema 4}\label{problema-4}}

El punto \(\mathbf{X} = (X, Y, Z)\) se distribuye uniformemente dentro
de una esfera de radio 1 alrededor del origen. Hallar la probabilidad de
los siguientes eventos:

\begin{enumerate}
\def\labelenumi{\arabic{enumi}.}
\tightlist
\item
  \(\mathbf{X}\) está dentro de una esfera de radio \(r\), \(r >0\).
\item
  \(\mathbf{X}\) está dentro de un cubo de longitud \(2/\sqrt{3}\)
  centrado alrededor del origen.
\item
  Todas las componentes de \(\mathbf{X}\) son positivas.
\item
  \(Z\) es negativa.
\item
  Hallar la distribución marginal de \(Y\) y \(Z\).
\item
  Hallar la distribución marginal de \(Y\).
\item
  Hallar la distribución condicional de \(X\) e \(Y\) dada \(Z\).
\item
  ¿Son independientes las variables \(X\), \(Y\) y \(Z\)?
\item
  Hallar los valores esperados y la matriz de covarianzas de la variable
  \((X,Y,Z)\).
\end{enumerate}

\hypertarget{soluciuxf3n-3}{%
\subsubsection{Solución}\label{soluciuxf3n-3}}

Indicaciones para la resolución:

La función de densidad conjunta será de la forma \(f_X(x,y,z)=c\), si
\((x,y,z)\) está en la esfera de radio 1 centrada en el origen y
\(f_X(x,y,z)=0\), en caso contrario. Para hallar \(c\), hay que imponer
que la integral triple de la función \(f_X\) tiene que ser \(1\) en la
esfera anterior. Dicha integral valdrá

\[c\cdot Volumen(esfera)=c\cdot (4/3)\cdot \pi=1, de donde se deduce que c=1/((4/3)\cdot \pi)=3/(4\cdot \pi).\]

Hallemos las probabilidades pedidas:

La probabilidad de que X esté en una esfera de radio \(r\), será \(1\)
si \(r\) es mayor que \(1\) ya que la esfera de radio \(r\) para \(r>1\)
incluye la esfera de radio \(1\) y valdrá
\(c\cdot 4\cdot \pi\cdot r^3/3\) si \(r<1\), ya que la integral sobre un
esfera más pequeña de la función de densidad conjunta sería
\(c\cdot \mbox{Volumen(esfera radio r)}\) y por tanto, la probabilidad
valdrá: \(r^3\).

El cubo considerado sería
\((-1/\sqrt{3},1/\sqrt{3})x( (-1/\sqrt{3},1/\sqrt{3})x (-1/\sqrt{3},1/\sqrt{3})\).
Los vértices del cubo estarían en la superficie de la esfera ya que
\((1/\sqrt{3})^2+(1/\sqrt{3})^2+(1/\sqrt{3})^2=1\). En pocas palabras,
el cubo estaría incluído en la esfera. La probabilidad pedida sería
entonces:

\[c\cdot \mbox{Volumen(cubo)}=3/(4\cdot\pi)\cdot (2/\sqrt{3})^3=2/(\pi\cdot \sqrt{3}).\]

La probabilidad de que todas las componentes de X sean positivas sería
equivalente a calcular el volumen del octante donde X,Y y Z son
positivos. Dejamos los detalles para vosotros.

La probabilidad de que Z sea negativa sería equivalente a calcular el
volumen de la media esfera del hemisferio sur.

Para hallar la distribución marginal de \((Y,Z)\), primero tenemos que
proyectar la esfera sobre el plano \((Y,Z)\) y ver qué figura 2D sale.
Si os fijáis, saldría el círculo de centro \((0,0)\) y radio 1. Ahora
para cada valor (y,z) dentro del círculo hay que hallar la integral para
todos los x tal que (x,y,z) esté en la esfera de centro \((0,0,0)\) y
radio \(1\) de la función constante \(c\). Dicha integral sería
\(c\cdot \mbox{longitud del segmento donde están las } x\).

Para hallar la distribución marginal de \(Y\), hay que fijarse que los
valores de y para los que f\_Y no se anula sería el intervalo
\((-1,1)\). Entonces para cada valor y entre \(-1\) y \(1\) hay que
hacer la integral doble de f\_X(x,y,z) para todos los (x,z) tal que
\((x,y,z)\) esté en la esfera de centro \((0,0,0)\) y radio \(1\) de la
función constante \(c\). Dicha integral sería
\(c\cdot \mbox{área de la región de los } (x,z)\).

Para hallar la distribución condicional de \((X,Y)\) dado \(Z=z\) hay
que usar la fórmula \(f_X(x,y,z)/f_Z(z)\), calculando previamente
\(f_Z(z)\) cuya expresión será la misma que \(f_Y(y)\) cambiando \(y\)
por \(z\) e \(Y\) por \(Z\) por simetría.

Ver si \(X\), \(Y\) y \(Z\) son independientes es equivalente a
comprobar si \(f_X(x,y,z)=f_X(x)\cdot f_Y(y)\cdot f_Z(z)\), para todos
los valores \(x,y,z\).

Para hallar los valores esperados de cada variable lo podéis hacer de
dos formas: integrando las funciones de densidad marginales o integrando
la función de densidad conjunta. Por ejemplo, para hallar \(E(X)\)
podéis hacer integral desde ``donde toque'' hasta ``donde toque'' de
\(x\cdot f_X(x)\) o integral triple de la función \(x\cdot f_X(x,y,z)\)
sobre la esfera centrada en el origen de radio 1.

Para hallar la matriz de covarianzas, tenéis que calcular previamente
los valores \(E(X\cdot Y)\), \(E(Y\cdot Z)\) y \(E(X\cdot Z)\). Yo lo
haría directamente a partir de la función de densidad conjunta. A partir
de dichos valores, las covarianzas son sencillas de calcular.

\hypertarget{problema-5.}{%
\subsection{Problema 5.}\label{problema-5.}}

Sea la variable \(3\) dimensional \((X, Y, Z)\) con función de densidad
conjunta: \[
f_{XYZ}(x,y,z)=\begin{cases}
k(x+y+z), & \mbox{si }0\leq x\leq 1,\ 0\leq y\leq 1,\ 0\leq z\leq 1,\\
0, & \mbox{en caso contrario.}
\end{cases}
\] 1. Hallar \(k\). 1. Hallar \(f_X(x)\), \(f_Y(y)\) y \(f_Z(z)\). 1.
Calcular matriz de covarianzas de \((X,Y,Z)\).

\hypertarget{soluciuxf3n-4}{%
\subsection{Solución}\label{soluciuxf3n-4}}

Indicaciones para la resolución:

Fijaos que el dominio donde la función de densidad conjunta no se anula
es el cuadrado unidad \([0,1]x[0,1]x[0,1]\).

Para hallar \(k\), hay que integrar la función \(k(x+y+z)\) en el
cuadrado anterior e imponer que dicha integral debe valer \(1\). Los
límites de integración para cada una de las variables serían \(0\) y
\(1\). Por tanto, la integral triple es sencilla de calcular y no
tendría demasiada dificultad hallar \(k\).

Para hallar las marginales, fijaos que por simetría basta hallar una, ya
que las demás tendrían la misma expresión. Por ejemplo, para hallar
\(f_X(x)\), hay que fijarse que \(x\) tiene que estar entre \(0\) y
\(1\) para que \(f_X(x)\) no se anule. El valor de \(f_X(x)\) sería la
integral doble de \(f_{XYZ}(x,y,z)\) para los \(y\), \(z\)
pertenecientes al cuadrado unidad \([0,1]x[0,1]\). Os dejamos los
detalles.

Para el ccáculo de la matriz de covarianzas utilizad los comentarios del
problema anterior pero usando la correspondiente función de densidad
conjunta y el correspondiente dominio donde dicha función no se anula
(cubo unidad).

\hypertarget{independencia-de-variables-aleatorias}{%
\section{Independencia de variables
aleatorias}\label{independencia-de-variables-aleatorias}}

\hypertarget{problema-6.}{%
\subsection{Problema 6.}\label{problema-6.}}

Supongamos que las variables aleatorias \(X\), \(Y\) y \(Z\) son
independientes. Hallar las probabilidades siguientes en términos de
\(F_X\), \(F_Y\) y \(F_Z\): 1. \(P(|X|<5,\ Y<4,\ Z^3>8)\). 1.
\(P(X=5,\ Y>0,\ Z>1)\). 1. \(P(\min(X,Y,Z)<2)\). 1.
\(P(\max(X, Y, Z)>6)\).

\hypertarget{momentos}{%
\section{Momentos}\label{momentos}}

\hypertarget{problema-7.}{%
\subsection{Problema 7.}\label{problema-7.}}

Hallar los valores esperados y la matriz de covarianzas para los
problemas 1 y 2 de la sección de variables aleatorias multidimensionales
continuas.

\hypertarget{variable-aleatoria-normal-multidimensional}{%
\section{Variable aleatoria normal
multidimensional}\label{variable-aleatoria-normal-multidimensional}}

\hypertarget{problema-8.}{%
\subsection{Problema 8.}\label{problema-8.}}

Sea \(\mathbf{X}=(X_1,X_2,X_3)\) una variable normal \(3\)-dimensional
con vector de medias y matriz de covarianzas dadas por: \[
\mathbf{\mu}_{\mathbf{X}} =\begin{pmatrix}1\\0\\2\end{pmatrix},\quad \mathbf{M}_{\mathbf{X}} =\begin{pmatrix}3/2 & 0 & 1/2 \\ 0 & 1 & 0 \\ 1/2 & 0 & 3/2\end{pmatrix}.
\]

\begin{enumerate}
\def\labelenumi{\arabic{enumi}.}
\tightlist
\item
  Hallar la función de densidad conjunta para la variable
  \(\mathbf{X}\).
\item
  Hallar las distribuciones marginales de las variables \(X_1\), \(X_2\)
  y \(X_3\).
\item
  Hallar una transformación lineal \(\mathbf{A}\) tal que la variable
  aleatoria \(3\)-dimensional \(\mathbf{Y}=\mathbf{A}\mathbf{X}\)
  consiste en variables normales independientes.
\item
  Hallar la función de densidad conjunta para la variable
  \(\mathbf{Y}\).
\end{enumerate}

\hypertarget{soluciuxf3n-5}{%
\subsection{Solución:}\label{soluciuxf3n-5}}

Indicaciones par la resolución:

La función de densidad conjunta se calcula de forma sencilla utilizando
el vector de medias y la matriz de covarianzas dado.

Para hallar las marginales, fijaos que por ejemplo
\(X_1=(1,0,0)\cdot (X_1,X_2,X_3)^t=(1,0,0)\cdot \begin{pmatrix}X_1\\ X_2\\ X_3\end{pmatrix}\)
. Ahora usando la matrix fila \(C=(1,0,0)\) y aplicando la proposición
de los apuntes que nos dice cómo se transforma \(C\cdot X\), o sea,
cuáles son la media y la varianza de la variable \(X_1\), podemos hallar
los parámetros de las marginales \(X_1, X_2\) y \(X_3\).

Para resolver el tercer apartado, hay que calcular una matriz \(A\) tal
que la matriz de covarianzas de la variable \(Y = A\cdot X\) tiene que
ser diagonal ya que como las componentes de la variable \(Y\) deben ser
independientes, las covarianzas de dos de ellas cualesquiera debe ser
\(0\). La proposición de los apuntes que nos dice cómo se transforma
\(A\cdot X\) nos da la expresión de la matriz de covarianzas de la
variable 3-dimensional Y en función de \(A\) y de la matriz de
covarianzas de la variable 3-dimensional X. Hay que imponer que dicha
matriz de covarianzas de \(Y\) debe ser diagonal. Indicación: pensad en
los valores propios y vectores propios de la matriz de covarianzas de la
variable 3-dimensional \(X\).

Una vez hallada el vector de medias y la matriz de covarianzas de la
variable 3-dimensional \(Y\), hallar la función de densidad conjunta es
sencillo aplicando la definición de una normal 3-dimensional.

\hypertarget{problema-9.}{%
\subsection{Problema 9.}\label{problema-9.}}

Supongamos que \(X_1, X_2, X_3\) y \(X_4\) son variables aleatorias
normales independientes de media cero y varianza 1 que se procesan de la
siguiente manera: \[
Y_1 = X_1 + X_2,\  Y_2 = X_2 + X_3,\  Y_3 = X_3 + X_4.
\] 1. Hallar la matriz de covarianzas de la variable \(3\)-dimensional
\(\mathbf{Y} = (Y_1, Y_2, Y_3)\). 1. Hallar la función de densidad
conjunta para la variable \(\mathbf{Y}\). 1. Hallar la función de
densidad conjunta para \(Y_1\) e \(Y_2\) y para \(Y_1\) e \(Y_3\). 1.
Hallar una transformación \(\mathbf{A}\) tal que el vector
\(\mathbf{Z} = \mathbf{A}\mathbf{Y}\) consista en variables aleatorias
normales independientes.

\hypertarget{soluciuxf3n-6}{%
\subsubsection{Solución}\label{soluciuxf3n-6}}

Indicaciones para la resolución:

Para hallar la matriz de covarianzas de la variable 3-dimensional Y, hay
que hallar la matriz de transformación \(C\) tal que \(Y=C\cdot X\).
Luego, usando la proposición de los apuntes que nos da el vector de
medias y la matriz de covarianzas de la variable \(Y\) a partir de las
de \(X\) y la matriz de transformación \(C\), es bastante sencillo
hallar dicha matriz de covarianzas. Pensad que la matriz de covarianzas
de \(X\) es la matriz identidad 4x4 ya que nos dicen que las componentes
de la variable 4-dimensional \(X\) son independientes de media \(0\) y
varianza \(1\).

Una vez hallada la matriz de covarianzas y el vector de medias de la
variable 3-dimensional Y, basta aplicar la definición de una normal
3-dimensional para calcular la función de densidad conjunta.

Para hallar la función de densidad conjunta para \(Y_1\) e \(Y_2\),
mirar el ejemplo de los apuntes que calcula el vector de medias y la
matriz de covarianzas de las componentes \((X_2,X_4,X_5)\) de una
variable 5-dimensional normal \(X=(X_1,X_2,X_3,X_4,X_5)\). La técnica a
aplicar es la misma. Una vez hallado el vector de medias y la matriz de
covarianzas, hallar la función de densidad es sencillo como ya hemos
comentado.

Para hacer el último apartado, leeros cómo se hace el cuarto apartado
del problema anterior. Se hace de forma parecida.

\end{document}
